\documentclass[aspectratio=43]{beamer}
\usepackage[utf8]{inputenc}
\usepackage{graphicx}
\usepackage{amsfonts, amssymb, amsmath}
\usepackage{float}
\usepackage{fancyhdr}
\usepackage{geometry}  
\usepackage{lastpage} % obtenir le nombre total de page
\usepackage{titlesec} % permet de modifier la taille des sections 


\title{Comment modéliser avec une approche réaliste de l’eau dans un jeu vidéo ?}
\author{Fortier Benjamin - Posez Bastien}
\date{\today}

\fancypagestyle{landscape}{
    \fancyhf{}
    \renewcommand{\headrulewidth}{0.5pt}
    \renewcommand{\footrulewidth}{0.5pt} 
    \fancyfoot[R]{\thepage /\pageref{LastPage}}
    \fancyfoot[L]{\leftmark \, - \rightmark}
}

\begin{document}
\huge
\maketitle

\newpage
\thispagestyle{landscape}
\vspace*{2pt}
\textbf{Mise en contexte:}
\begin{center}
    L’industrie du jeu qui ne fait que de se développer, la présence de réalisme au sein de nouveaux jeux est un argument marketing important qui permet de faciliter l’immersion des joueurs. 
\end{center}




\end{document}